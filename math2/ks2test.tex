\documentclass[landscape,a3paper,8pt]{ltjsarticle}
\usepackage{luatexja}

%% ----------------------------------------------------------------------------------------------------
%% 版面設定
%% ----------------------------------------------------------------------------------------------------
\usepackage[top=18mm,bottom=18mm,left=4mm,right=4mm]{geometry}
\setlength{\columnsep}{1.8\zw}
\setlength{\columnseprule}{0.1pt}
\parindent=0pt


%% ----------------------------------------------------------------------------------------------------
%% Tikz関係+色
%% ----------------------------------------------------------------------------------------------------
\usepackage{xcolor}
\usepackage{tikz}
\usepackage[most]{tcolorbox}


%% ----------------------------------------------------------------------------------------------------
%% ヘッダー・フッター
%% ----------------------------------------------------------------------------------------------------
\usepackage{fancyhdr}
\pagestyle{fancy}
\fancyhead[RO,RE]{\colorbox{blue!10!white}{※3回チェックします。1回目は授業後の取り組み状況(2点)、2回目は宿題の取り組み状況(2点)、3回目は最終評価(2点)です。}}
\fancyfoot[RO,RE]{\textcolor{blue}{※青い枠が課題です。青い枠内で解いてください。} \textcolor{red}{※赤い枠も課題です。こちらは記述問題としてチェックします。途中もしっかりと残しておこう。} ※少しずつ〇付けをしていきます。1週間ぐらいしたら評価を行います。お早めに。}
\fancyfoot[CO,CE]{}
\newcommand{\myHeader}[1]{\fancyhead[LO,LE]{\colorbox{red!10!white}{\large\gtfamily\bfseries{}#1}}}


%% ----------------------------------------------------------------------------------------------------
%% 環境作成 by environ
%% ----------------------------------------------------------------------------------------------------
\usepackage{environ}
%% --------------------------------------------------------------------------------
%% ---------- \BODY で高さを自動調整する解答ボックス
\NewEnviron{ansBlockAuto}{%
 \begin{tcolorbox}[colframe=blue,colback=white]
  \BODY
 \end{tcolorbox}
}
%% ---------- 高さを指定した解答ボックス
\NewEnviron{ansBlockSize}[1]{%
 \begin{tcolorbox}[colframe=blue,colback=white,height=#1\zw]
  \BODY
 \end{tcolorbox}
}
%% ---------- 公式などを目立たせるボックス
\NewEnviron{thmBlock}{%
 \begin{tcolorbox}[colframe=red,colback=red!5!white]
  \BODY
 \end{tcolorbox}
}


%% ----------------------------------------------------------------------------------------------------
%% コマンド作成
%% ----------------------------------------------------------------------------------------------------
%% --------------------------------------------------------------------------------------教科書取り込み
\newcommand{\textbook}[2][1.00]{%
  \begin{minipage}{0.9\columnwidth}
    \includegraphics[trim=50 100 110 100, width=#1\textwidth, clip]{#2}
  \end{minipage}
}
\newcommand{\AnswerBox}[2][1.00]{%
  \vspace{.5zw}\\
  \begin{tikzpicture}%
    \draw[dotted,very thick,blue] (0,0) rectangle (#1\columnwidth,#2\zw);%
  \end{tikzpicture}
}
\newcommand{\ReturnBox}{%
    おつかれさまでした。\vspace{1\zw}\\
    \textcolor{blue}{\bfseries すべて取り組んだら今回の学びを「振り返り」ましょう。}\\
    \qquad{}・評価2点・・・記入できた。\\
    \qquad{}・評価2点・・・文章で書けた。\\
    \qquad{}・評価2点・・・2文以上書けた。\vspace{1\zw}\\
    \textcolor{red}{\bfseries Y.やったこと}\vspace{-.5\zw}\\
    {\footnotesize{}今回の課題で学んだことを自分のことばでまとめよう}\vspace{.5\zw}\\
    \begin{tikzpicture}
      \draw[ultra thick,red] (0,0) rectangle (0.95\columnwidth,5);
    \end{tikzpicture}
    \vspace{1\zw}
    \textcolor{red}{\bfseries W.わかったこと}\vspace{-.5\zw}\\
    {\footnotesize{}今回の課題で理解したことを自分のことばでまとめよう}\vspace{.5\zw}\\
    \begin{tikzpicture}
      \draw[ultra thick,red] (0,0) rectangle (0.95\columnwidth,5);
    \end{tikzpicture}
    \vspace{1\zw}
    \textcolor{red}{\bfseries T.次にやること}\vspace{-.5\zw}\\
    {\footnotesize{}次の課題に向けて何をすべきか自分のことばで残しましょう}\vspace{.5\zw}\\
    \begin{tikzpicture}
      \draw[ultra thick,red] (0,0) rectangle (0.95\columnwidth,5);
    \end{tikzpicture}
    \vfill
}



\usepackage{graphicx}%includegraphics
\usepackage{multicol}




%% ----------------------------------------------------------------------------------------------------
%% 数学関係
%% ----------------------------------------------------------------------------------------------------
\usepackage{emath}



\newcommand{\Rei}[1]{\fcolorbox{blue}{cyan!20!white}{\ 例#1\ }\vspace{.5\zw}\\}
\newcommand{\Reidai}[1]{\fcolorbox{green}{green!20!white}{\ 例題#1\ }\quad{}}
\newcommand{\Toi}[1]{\fcolorbox{red}{pink!20!white}{\ 問#1\ }\vspace{.5\zw}\\}
\newcommand{\Setumatu}[1]{\fcolorbox{black}{gray!20!white}{\ 節末#1\ }\quad{}}
\newcommand{\Libry}[1]{\fcolorbox{orange}{orange!20!white}{\ #1\ }\quad{}}

\newcommand{\RedBold}[1]{\textcolor{red}{\textbf{#1}}}

\begin{document}
% ============================================================================================================
\myHeader{第2章「統計的な推測」> 第1節「確率分布」> 4「確率変数の和と期待値」}
% ============================================================================================================
\begin{multicols*}{4}
 
 \textbook{text/ks2/ksB_053.PNG}
 \vfill
 
 \Rei{6}
 1,2の数が書かれたカードが,それぞれ5枚, 3枚ある。
 この8枚のカードから1枚を引き,カードに書かれた数を$X$とする。
 引いたカードを\textbf{もとに戻さずに}もう1回引き,カードに書かれた数を$Y$とする。
 
 \hfill\RedBold{※ ステップを踏みながら例6を理解しよう。}

 (1). 確率$P(X=1, Y=1)$を求めよう。
 
 \begin{ansBlockSize}{15}
  \fbox{$X=1$}  
  8枚のカードから1枚引いたときに1が出るのは$\bunsuu58$
  
  \fbox{$Y=1$}
  残り7枚のカードから1枚引いたときに1が出るのは$\bunsuu47$
  
  よって, $P(X=1, Y=1) = \bunsuu58 \times \bunsuu47 = \bunsuu{20}{56}$
  
  \hfill{\color{blue}\small%
  ※ 分布を考えるときは約分しない。
  }
 \end{ansBlockSize} 
 \vspace{1\zw}
 
 (2). 確率$P(X=1, Y=2)$を求めよう。
 
 \begin{ansBlockSize}{15}
 \end{ansBlockSize}

 \columnbreak{}
 \null
 \vfill
 
 (3). 確率$P(X=2, Y=1)$を求めよう。
 
 \begin{ansBlockSize}{15}
 \end{ansBlockSize}
 \vspace{1\zw}

 (4). 確率$P(X=2, Y=2)$を求めよう。
 
 \begin{ansBlockSize}{15}
 \end{ansBlockSize}
 \vspace{1\zw}
 
 (5). ここまでの結果を利用して,下の表を埋めよう。
 
 \begin{ansBlockAuto}
  \begin{tabular}{|p{3\zw}|p{3\zw}|p{3\zw}|p{3\zw}|}\hline
   & $Y=1$ & $Y=2$ & 計 \rule[-5mm]{0mm}{12mm}\\\hline
   $X=1$ & \ $\bunsuu{20}{56}$ & & \rule[-5mm]{0mm}{12mm}\\\hline
   $X=2$ & & & \rule[-5mm]{0mm}{12mm}\\\hline
   計 & & & \rule[-5mm]{0mm}{12mm}\\\hline
  \end{tabular}
  
  \hfill\RedBold{※ この表をXとYの\textbf{同時分布}という。}
 \end{ansBlockAuto}
 \vspace{1\zw}
 
 (6). $X$と$Y$の同時分布から,$X$,$Y$の確率分布を探して表を埋めよう。
 
 \begin{ansBlockAuto}
  \begin{tabular}{|c|c|c|c|}\hline
   $X$ & 1 & 2 & 計\rule[-3mm]{0mm}{8mm}\\\hline
   $p$ & \hspace{3\zw} & \hspace{3\zw} & \hspace{2\zw} \rule[-5mm]{0mm}{12mm}\\\hline
  \end{tabular}
  \hfill
  \begin{tabular}{|c|c|c|c|}\hline
   $Y$ & 1 & 2 & 計\rule[-3mm]{0mm}{8mm}\\\hline
   $p$ & \hspace{3\zw} & \hspace{3\zw} & \hspace{2\zw} \rule[-5mm]{0mm}{12mm}\\\hline
  \end{tabular}
  
  \RedBold{※ $X$の周辺分布$=X$の確率分布}\hfill\RedBold{※ $Y$の周辺分布$=Y$の確率分布}
 \end{ansBlockAuto}
 
 \columnbreak{}
 
 \textbook{text/ks2/ksB_054.PNG}
 \textbook{text/ks2/ksB_055.PNG}
 
 \columnbreak{}
 
 \Rei{7}
 表に2または10, 裏に3または6の数が書かれたカードが13枚あり,その表と裏の内訳は,次の表のようになっているとする。
 \vspace{.5\zw}
 
 \hspace{2\zw}
 (表)
 \begin{tabular}{|c|c|c|c|}\hline
  数 & 2 & 10 & 計\\\hline
  枚数 & 6 & 7 & 13\\\hline
 \end{tabular}
 \hfill
 (裏)
 \begin{tabular}{|c|c|c|c|}\hline
  数 & 3 & 6 & 計\\\hline
  枚数 & 8 & 5 & 13\\\hline
 \end{tabular}
 \hfill
 \vspace{.5\zw}

 この13枚のカードの中から1枚を引くとき,表に書かれた数$X$と裏に書かれた数$Y$の和$X+Y$の期待値を求めてみよう。

 \hfill
 \RedBold{※ ステップを踏みながら例7を理解しよう。}

 (1). $X+Y$の取り得る値を下の表を使って求めよう。
 
 \begin{ansBlockAuto}
  \begin{tabular}{|c|c|c|}\hline
   & $Y=3$ & $Y=6$\rule[-5mm]{0mm}{11mm}\\\hline
   $X=2$ & \hspace{2\zw} & \hspace{2\zw} \rule[-5mm]{0mm}{11mm}\\\hline
   $X=10$ & 13 & \rule[-5mm]{0mm}{11mm}\\\hline
  \end{tabular}
  \quad\RedBold{※ よって, $X+Y = 5, 8, 13, 16$}
 \end{ansBlockAuto}
 \vspace{1\zw}
 
 (2). 下の表を利用して$X+Y$の確率分布を求めよう。
 
 \begin{ansBlockAuto}
  \RedBold{と思ったけど,何もわかりません。orz}
  \vspace{.5\zw}

  \begin{tabular}{|c|c|c|c|c|c|}\hline
   $X+Y$ & 5 & 8 & 13 & 16 & 計 \\\hline
   $p$ & \hspace{2\zw} & \hspace{2\zw} & \hspace{2\zw} & \hspace{2\zw} & 1 \rule[-4mm]{0mm}{10mm}\\\hline
  \end{tabular}
  
  {\color{blue}\small%
  例えば,カードは全部で13通り。
  
  $X+Y=5$になるのは$X=2$,$Y=3$のときだけ。
  
  よって表が2,裏が3のカードを数えればいいのですが・・・何枚あるか分かりますか。まったく分かりません。つまり$P(X+Y=5)$は求められない!
  }
  
  \RedBold{この方法では$E(X+Y)$を求めることはできません。}

  {\color{blue}\small%
  条件を満たす13枚のカードを実際に作れば求められます。が・・・大変です。
  }
 \end{ansBlockAuto}
 
 \begin{tcolorbox}[colframe=red,colback=pink!10!white]
  そこで {\large\color{red} $E(X+Y) = E(X)+E(Y)$} の登場です!
 \end{tcolorbox}
 
 (3). 下の表を埋めて確率分布を作り, 期待値$E(X)$, $E(Y)$を求めよう。
 
 \begin{ansBlockSize}{16}
  \begin{minipage}{0.5\columnwidth}
   \begin{tabular}{|c|c|c|c|}\hline
    $X$ & 2 & 10 & 計\\\hline
    $p$ & \hspace{2\zw} & \hspace{2\zw} & 1\rule[-4mm]{0mm}{9mm}\\\hline
   \end{tabular}
   \vspace{.5\zw}\\
   $E(X)=$
  \end{minipage}
  \begin{minipage}{0.5\columnwidth}
   \begin{tabular}{|c|c|c|c|}\hline
    $Y$ & 3 & 6 & 計\\\hline
    $p$ & \hspace{2\zw} & \hspace{2\zw} & 1\rule[-4mm]{0mm}{9mm}\\\hline
   \end{tabular}
   \vspace{.5\zw}\\
   $E(Y)=$
  \end{minipage}
 \end{ansBlockSize}
 \vspace{1\zw}
 
 (4). $X+Y$の期待値$E(X+Y)$を求めよう。
 
 \begin{ansBlockSize}{8}
  $E(X+Y)=E(X)+E(Y)$より
 \end{ansBlockSize}

 \columnbreak{}
 
 \textbook{text/ks2/ksB_055.PNG}
 \vfill
 
 \Toi{8}
 1個のサイコロを2回投げるとき,出る目の和の期待値を求めよ。
 
 \RedBold{※ 数学A風に解いてみよう。}
 
 (1). 下の表を埋めよう。また,出る目の和$Z$のとり得る値を求めよう。
 
 \begin{ansBlockAuto}
  \begin{tabular}{|c|c|c|c|c|c|c|}\hline
   和& 1 & 2 & 3 & 4 & 5 & 6 \\\hline
   1 & & & & & & \\ \hline
   2 & & & & & & \\ \hline
   3 & & & & & & \\ \hline
   4 & & & & & & \\ \hline
   5 & & & & & & \\ \hline
   6 & & & & & & \\ \hline
  \end{tabular}
  \quad$Z=$
 \end{ansBlockAuto}
 
 (2). 下の表を埋めて出る目の和$Z$の確率分布を作ろう。
 
 \begin{ansBlockAuto}
  \begin{tabular}{|c|c|c|c|c|c|c|c|c|c|c|c|c|}\hline
   $Z$ & 2 & 3 & 4 & 5 & 6 & 7 & 8 & 9 & 10 & 11 & 12 & 計 \\ \hline
   $p$ & & & & & & & & & & & & \rule[-5mm]{0mm}{11mm}\\ \hline
  \end{tabular}
 \end{ansBlockAuto}
 
 (3). 出る目の和の期待値を求めよう。
 
 \begin{ansBlockSize}{8}
 \end{ansBlockSize}
 
 \columnbreak{}
 
 \begin{thmBlock}
  和の期待値\\
  \hspace{5\zw}$E(X+Y)=E(X)+E(Y)$
 \end{thmBlock}
 $X+Y$の期待値が欲しいなら,$X$,$Y$の期待値を求めればいい!
 \vfill
 
 \Toi{8}
 1個のサイコロを2回投げるとき,出る目の和の期待値を求めよ。
 
 \RedBold{※ 和の期待値の公式を使ってみよう。}
 \vspace{1\zw}
 
 (1). サイコロ1個目を投げるとき,出る目$X$の期待値を求めよう。
 
 \begin{ansBlockSize}{15}
  \begin{tabular}{|c|c|c|c|c|c|c|c|}\hline
   $X$ & 1 & 2 & 3 & 4 & 5 & 6 & 計 \\ \hline
   $p$ & \hspace{2\zw} & \hspace{2\zw} & \hspace{2\zw} & \hspace{2\zw} & \hspace{2\zw} & \hspace{2\zw} & \hspace{2\zw} \rule[-5mm]{0mm}{11mm}\\ \hline
  \end{tabular}
  \vspace{.5\zw}
  
  $E(X) = $
 \end{ansBlockSize}
 \vspace{1\zw}
 
 (2). サイコロ2個目を投げるとき,出る目$Y$の期待値を求めよう。
 
 \begin{ansBlockSize}{15}
  \begin{tabular}{|c|c|c|c|c|c|c|c|}\hline
   $Y$ & 1 & 2 & 3 & 4 & 5 & 6 & 計 \\ \hline
   $p$ & \hspace{2\zw} & \hspace{2\zw} & \hspace{2\zw} & \hspace{2\zw} & \hspace{2\zw} & \hspace{2\zw} & \hspace{2\zw} \rule[-5mm]{0mm}{11mm}\\ \hline
  \end{tabular}
  \vspace{.5\zw}
  
  $E(Y) = $
 \end{ansBlockSize}
 \vspace{1\zw}
 
 (3). 2個のサイコロを投げるとき, 出る目の和$X$,$Y$の期待値を求めよう。
 
 \begin{ansBlockSize}{10}
  $E(X+Y) = E(X) + E(Y)$より
 \end{ansBlockSize}
 
 \columnbreak{}
 
 \begin{thmBlock}
  和の期待値\\
  \hspace{5\zw}$E(X+Y+Z)=E(X)+E(Y)+E(Z)$
 \end{thmBlock}
 ・・・つまり何個でもOK!
 \vfill
 
 \Toi{9}
 500円硬貨1枚, 100円硬貨1枚, 10円硬貨1枚を投げるとき,
 表が出た硬貨の金額の和の期待値を求めよ。
 
 \RedBold{※確率変数が何個になっても和の期待値の公式が使えます♪}
 
 (1). 500円硬貨1枚を投げるとき, 表が出た硬貨の金額$X$の期待値$E(X)$を求めよ。
 
 \begin{ansBlockSize}{15}
  \begin{tabular}{|c|c|c|c|}\hline
   $X$ & 0 & 500 & 計 \\ \hline
   $p$ & \hspace{2\zw} & \hspace{2\zw} & \hspace{2\zw} \rule[-5mm]{0mm}{11mm}\\ \hline
  \end{tabular}
  \vspace{.5\zw}\\
  $E(X) = $
 \end{ansBlockSize}

 (2). 100円硬貨1枚を投げるとき, 表が出た硬貨の金額$Y$の期待値$E(Y)$を求めよ。
 
 \begin{ansBlockSize}{15}
  \begin{tabular}{|c|c|c|c|}\hline
   $Y$ & 0 & 100 & 計 \\ \hline
   $p$ & \hspace{2\zw} & \hspace{2\zw} & \hspace{2\zw} \rule[-5mm]{0mm}{11mm}\\ \hline
  \end{tabular}
  \vspace{.5\zw}\\
  $E(Y) = $
 \end{ansBlockSize}
 
 (3). 10円硬貨1枚を投げるとき, 表が出た硬貨の金額$Z$の期待値$E(Z)$を求めよ。
 
 \begin{ansBlockSize}{15}
  \begin{tabular}{|c|c|c|c|}\hline
   $Z$ & 0 & 10 & 計 \\ \hline
   $p$ & \hspace{2\zw} & \hspace{2\zw} & \hspace{2\zw} \rule[-5mm]{0mm}{11mm}\\ \hline
  \end{tabular}
  \vspace{.5\zw}\\
  $E(Z) = $
 \end{ansBlockSize}
 
 (4). $E(X+Y+Z)$を求めよ。
 
 \begin{ansBlockSize}{15}
  $E(X+Y+Z) = E(X)+E(Y)+E(Z)$より
 \end{ansBlockSize}

 \columnbreak{}
 
 \ReturnBox{}
 
\end{multicols*}
\end{document}

