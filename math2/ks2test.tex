\documentclass[landscape,a3paper,8pt]{ltjsarticle}
\usepackage{luatexja}


%% ----------------------------------------------------------------------------------------------------
%% 版面設定
%% ----------------------------------------------------------------------------------------------------
\usepackage[top=18mm,bottom=18mm,left=4mm,right=4mm]{geometry}
\setlength{\columnsep}{1.8\zw}
\setlength{\columnseprule}{0.1pt}
\parindent=0pt


%% ----------------------------------------------------------------------------------------------------
%% Tikz関係+色
%% ----------------------------------------------------------------------------------------------------
\usepackage{xcolor}
\usepackage{tikz}
\usepackage[most]{tcolorbox}


%% ----------------------------------------------------------------------------------------------------
%% ヘッダー・フッター
%% ----------------------------------------------------------------------------------------------------
\usepackage{fancyhdr}
\pagestyle{fancy}
\fancyhead[RO,RE]{\colorbox{blue!10!white}{※3回チェックします。1回目は授業後の取り組み状況(各2点)、2回目は宿題の取り組み状況(各2点)、3回目は最終評価(各2点)です。}}
\fancyfoot[RO,RE]{\textcolor{blue}{※青い枠が課題です。青い枠内で解いてください。} \textcolor{red}{※赤い枠も課題です。こちらは記述問題としてチェックします。途中もしっかりと残しておこう。} ※少しずつ〇付けをしていきます。1週間ぐらいしたら評価を行います。お早めに。}
\fancyfoot[CO,CE]{}
\newcommand{\myHeader}[1]{\fancyhead[LO,LE]{\colorbox{red!10!white}{\large\gtfamily\bfseries{}#1}}}


%% ----------------------------------------------------------------------------------------------------
%% 環境作成 by environ
%% ----------------------------------------------------------------------------------------------------
\usepackage{environ}
%% --------------------------------------------------------------------------------
%% ---------- \BODY で高さを自動調整する解答ボックス
\NewEnviron{ansBlockAuto}{%
 \begin{tcolorbox}[colframe=blue,colback=white]
  \BODY
 \end{tcolorbox}
}
%% ---------- 高さを指定した解答ボックス
\NewEnviron{ansBlockSize}[1]{%
 \begin{tcolorbox}[colframe=blue,colback=white,height=#1\zw]
  \BODY
 \end{tcolorbox}
}
%% ---------- 公式などを目立たせるボックス
\NewEnviron{thmBlock}{%
 \begin{tcolorbox}[colframe=red,colback=red!5!white]
  \BODY
 \end{tcolorbox}
}


%% ----------------------------------------------------------------------------------------------------
%% コマンド作成
%% ----------------------------------------------------------------------------------------------------
%% --------------------------------------------------------------------------------------教科書取り込み
\newcommand{\textbook}[2][1.00]{%
  \begin{minipage}{0.9\columnwidth}
    \includegraphics[trim=50 100 110 100, width=#1\textwidth, clip]{#2}
  \end{minipage}
}
\newcommand{\AnswerBox}[2][1.00]{%
  \vspace{.5zw}\\
  \begin{tikzpicture}%
    \draw[dotted,very thick,blue] (0,0) rectangle (#1\columnwidth,#2\zw);%
  \end{tikzpicture}
}
\newcommand{\ReturnBox}{%
    おつかれさまでした。\vspace{1\zw}\\
    \textcolor{blue}{\bfseries すべて取り組んだら今回の学びを「振り返り」ましょう。}\\
    \qquad{}・評価2点・・・記入できた。\\
    \qquad{}・評価2点・・・文章で書けた。\\
    \qquad{}・評価2点・・・2文以上書けた。\vspace{1\zw}\\
    \textcolor{red}{\bfseries Y.やったこと}\vspace{-.5\zw}\\
    {\footnotesize{}今回の課題で学んだことを自分のことばでまとめよう}\vspace{.5\zw}\\
    \begin{tikzpicture}
      \draw[ultra thick,red] (0,0) rectangle (0.95\columnwidth,5);
    \end{tikzpicture}
    \vspace{1\zw}
    \textcolor{red}{\bfseries W.わかったこと}\vspace{-.5\zw}\\
    {\footnotesize{}今回の課題で理解したことを自分のことばでまとめよう}\vspace{.5\zw}\\
    \begin{tikzpicture}
      \draw[ultra thick,red] (0,0) rectangle (0.95\columnwidth,5);
    \end{tikzpicture}
    \vspace{1\zw}
    \textcolor{red}{\bfseries T.次にやること}\vspace{-.5\zw}\\
    {\footnotesize{}次の課題に向けて何をすべきか自分のことばで残しましょう}\vspace{.5\zw}\\
    \begin{tikzpicture}
      \draw[ultra thick,red] (0,0) rectangle (0.95\columnwidth,5);
    \end{tikzpicture}
    \vfill
}



\usepackage{graphicx}%includegraphics
\usepackage{multicol}




%% ----------------------------------------------------------------------------------------------------
%% 数学関係
%% ----------------------------------------------------------------------------------------------------
\usepackage{emath}



\newcommand{\Rei}[1]{\fcolorbox{blue}{cyan!20!white}{\ 例#1\ }\vspace{.5\zw}\\}
\newcommand{\Reidai}[1]{\fcolorbox{green}{green!20!white}{\ 例題#1\ }\quad{}}
\newcommand{\Toi}[1]{\fcolorbox{red}{pink!20!white}{\ 問#1\ }\vspace{.5\zw}\\}
\newcommand{\Setumatu}[1]{\fcolorbox{black}{gray!20!white}{\ 節末#1\ }\quad{}}
\newcommand{\Libry}[1]{\fcolorbox{orange}{orange!20!white}{\ #1\ }\vspace{.5\zw}\\}

\newcommand{\RedBold}[1]{\textcolor{red}{\textbf{#1}}}

\renewcommand{\labelenumi}{(\arabic{enumi}).\ }

\newcommand{\ansKuuran}[1]{%
\ \tikz[baseline=(T.base)]
\node[draw=green, fill=green!5, text=green!10] (T) {\Large\ #1\ \null};
\ 
}

\usepackage{amsmath}	% required for `\align*' (yatex added)

\begin{document}
% ============================================================================================================
\myHeader{第5章「微分と積分」> 第3節「積分」> 不定積分・定積分そして面積}
% ============================================================================================================
\begin{multicols*}{4}
 
 \textbook{text/ks5/ks2_190.PNG}
 \textbook{text/ks5/ks2_191.PNG}
 
 \columnbreak{}
 
 \textbook{text/ks5/ks2_192.PNG}
 \vfill
 
 \RedBold{空欄を埋めて重要キーワードを確認しよう!}
 
 \begin{itemize}
  \item 関数$f(x)$に対して,
	微分すると $f(x)$ になる関数$F(x)$ を,\\
	\hspace{3\zw}$f(x)$の\ansKuuran{原始関数}という。
	\vfill
	\begin{itemize}
	 \item $F(x)$ を微分すれば $f(x)$ になる。
	 \item つまり・・・$F'(x) = $\ansKuuran{$f(x)$}
	 \item 原始関数は無数にある。その違いは定数部分のみであり, これらをまとめて$C$と表す。
	 \item $F(x)+C$をまとめて$f(x)$の\ansKuuran{不定積分}といい,\\
	       \hspace{3\zw}\ansKuuran{$\displaystyle\int f(x)\,dx$}で表す。
	\end{itemize}
	\vfill
  \item $f(x)$の不定積分を求めることを, $f(x)$を \ansKuuran{積分する}といい,\\
	\hspace{3\zw}定数$C$を\ansKuuran{積分定数}という。
	\begin{itemize}
	 \item とくに断らなくても, $C$は積分定数を表す。
	\end{itemize}
	\vfill
 \end{itemize}
 
 \begin{thmBlock}
  \RedBold{\fbox{積分する}} $\maru1$ $x$増やす $\maru2$ 増やした数で割る$\maru3$ 必要なら約分 $\maru4$ $+C$
  
  \hfill{}※ 微分も積分も「展開」してから!
  
  \fbox{微分する} $\maru1$ $x$の係数と指数をかける $\maru2$ $x$減らす $\maru3$ 定数は消す
 \end{thmBlock}
 
 \columnbreak{}
 
 \Rei{14,15}
 次の不定積分を求めよ。
 
 \begin{enumerate}
  \item $\displaystyle\int (3x^2-4x+2)\,dx$
	\vfill
  \item $\displaystyle\int (x-1)(x-2)\,dx$
	\vfill
 \end{enumerate}
 
 \RedBold{上の例を参考に不定積分の手順を確認しよう!}
 \vspace{1\zw}
 
 \Rei{14,15}
 次の不定積分を求めよ。
 
 \begin{enumerate}
  \item $\displaystyle\int (3x^2-4x+2)\,dx$
	\begin{ansBlockSize}{15}
	\end{ansBlockSize}
  \item $\displaystyle\int (x-1)(x-2)\,dx$
	\begin{ansBlockSize}{18}
	\end{ansBlockSize}
 \end{enumerate}
 
 \columnbreak{}
 
 \Toi{23}
 次の不定積分を求めよ。
 
 \begin{enumerate}
  \item $\displaystyle\int 7\,dx$
	\begin{ansBlockSize}{10}
	\end{ansBlockSize}
  \item $\displaystyle\int (6x^2+x-5)\,dx$
	\begin{ansBlockSize}{10}
	\end{ansBlockSize}
  \item $\displaystyle\int (x^3+4)\,dx$
	\begin{ansBlockSize}{10}
	\end{ansBlockSize}
  \item $\displaystyle\int (x+1)(x+3)\,dx$
	\begin{ansBlockSize}{15}
	\end{ansBlockSize}
  \item $\displaystyle\int (3t+2)^2\,dt$
	\begin{ansBlockSize}{15}
	\end{ansBlockSize}
 \end{enumerate}
 
 \columnbreak{}
 
 \Libry{Axis 420}
 次の不定積分を求めよ。
 
 \begin{enumerate}
  \item $\displaystyle\int (-5)\,dx$
	\begin{ansBlockSize}{6}
	\end{ansBlockSize}
  \item $\displaystyle\int (2x-1)\,dx$
	\begin{ansBlockSize}{6}
	\end{ansBlockSize}
  \item $\displaystyle\int (x^2-3x-2)\,dx$
	\begin{ansBlockSize}{6}
	\end{ansBlockSize}
  \item $\displaystyle\int (4x^3-3x^2-2x-1)\,dx$
	\begin{ansBlockSize}{6}
	\end{ansBlockSize}
 \end{enumerate}
 
 \Libry{Axis 421}
 次の不定積分を求めよ。
 
 \begin{enumerate}
  \item $\displaystyle\int (x-1)(x+2)\,dx$
	\begin{ansBlockSize}{10}
	\end{ansBlockSize}
  \item $\displaystyle\int (2x+3)^2\,dx$
	\begin{ansBlockSize}{10}
	\end{ansBlockSize}
  \item $\displaystyle\int (x+1)^3\,dx$
	\begin{ansBlockSize}{10}
	\end{ansBlockSize}
 \end{enumerate}
 
 \columnbreak{}
 
 \textbook{text/ks5/ks2_193.PNG}
 \vfill
 
 \RedBold{空欄を埋めて重要キーワードを確認しよう!}
 
 \begin{itemize}
  \item $f(x)$の原始関数の1つを$F(x)$とすると,\\
	\begin{itemize}
	 \item $F(b)-F(a)$は, 原始関数の選び方に関係なく\ansKuuran{定まる}。
	 \item $F(b)-F(a)$を, 関数$f(x)$の$a$から$b$までの\ansKuuran{定積分}という。
	 \item \ansKuuran{$\displaystyle\int{a}^{b}f(x)\,dx$}$=\left[F(x)\right]_{a}^{b} = F(b)-F(a)$
	 \item $a$をこの定積分の\ansKuuran{下端}, $b$を\ansKuuran{上端}という。
	 \item この定積分を求めることを\\
	       \hspace{3\zw}$f(x)$を$a$から$b$まで\ansKuuran{積分する}という。
	\end{itemize}
  \item 積分にも\ansKuuran{線形性}あり!
 \end{itemize}
 \vspace{1\zw}
 
 \Rei{16}
 $\displaystyle\int_{1}^{4}x^2\,dx$
 \vfill
 \vfill
 \vfill
 \null
 
 \columnbreak{}
 
 \textbook{text/ks5/ks2_194.PNG}
 \vspace{1\zw}
 
 \RedBold{※ 線形性を意識すれば覚えることが減る!}
 \vspace{1\zw}
 
 \Rei{17}
 $\displaystyle\int_{-1}^{3}(2t^2-5t)\,dt$
 \vfill
 \vfill
 \null
 
 \columnbreak{}
 
 \Toi{25}
 次の定積分を求めよ。
 \begin{enumerate}
  \item $\displaystyle\int_{1}^{2}(x-1)(x-2)\,dx$
	\begin{ansBlockSize}{24}
	\end{ansBlockSize}
	\vfill
  \item $\displaystyle\int_{3}^{0}(1-2t^2)\,dt$
	\begin{ansBlockSize}{24}
	\end{ansBlockSize}
	\vfill
  \item $\displaystyle\int_{1}^{-1}(y^2+2y+1)\,dy$
	\begin{ansBlockSize}{24}
	\end{ansBlockSize}
 \end{enumerate}
 
 \columnbreak{}
 
 \textbook{text/ks5/ks2_198.PNG}
 \textbook{text/ks5/ks2_199.PNG}
 
 \columnbreak{}
 
 \textbook{text/ks5/ks2_200.PNG}
 \textbook{text/ks5/ks2_201.PNG}
 
 \columnbreak{}
 
 \begin{tcolorbox}[title=The 面積の公式!,colback=orange!5,colframe=orange,colbacktitle=orange!10,coltitle=orange]
  $\maru1$\ 簡単なグラフをかいて求めたい面積を確認!
  
  $\maru2$\ $x$座標の\textbf{左}端と\textbf{右}端、グラフの\textbf{上下}を確認して
  
  $\maru3$\ 公式にセット!\\
  \LARGE
  面積 $= \displaystyle\int_{左}^{右} \left( 上 - 下 \right)\,dx$
 \end{tcolorbox}
 \vspace{1\zw}
 
 \RedBold{面積の公式はこれ1つで十分!グラフは上下左右が分かる簡単なものでOK!}
 \vspace{1\zw}
 
 \Rei{23}
 放物線$y=x^2-1$と直線$y=x+1$および2直線$x=0$, $x=1$で囲まれた部分の面積$S$を求めてみよう。
 \vfill
 
 \Toi{35}
 放物線$y=x^2-3x$と直線$y=x+5$および2直線$x=1$, $x=3$で囲まれた部分の面積$S$を求めよ。
 \begin{ansBlockSize}{40}
 \end{ansBlockSize}
 
 \columnbreak{}
 \Reidai{12}
 次の2つの放物線で囲まれた部分の面積$S$を求めよ。\\
 \hspace{3\zw}$y=x^2+2x$,\quad{}$y=-x^2+4$
 \vfill
 
 \Toi{36}
 次の2つの放物線で囲まれた部分の面積$S$を求めよ。\\
 \hspace{3\zw}$y=x^2-1$,\quad{}$y=-x^2+x$
 \begin{ansBlockSize}{45}
 \end{ansBlockSize}
 
 \columnbreak{}
 
 \Rei{21}
 放物線$y=x^2+1$と$x$軸および2直線$x=1$, $x=3$で囲まれた部分の面積$S$を求めてみよう。
 \begin{ansBlockSize}{40}
  グラフは右のようになるから\\
  $S = \displaystyle\int_{1}^{3}\left\{(x^2+1)-0\right\}\,dx$
 \end{ansBlockSize}
 
 \Toi{33}
 放物線$y=6x-2x^2$と$x$軸および2直線$x=1$, $x=2$で囲まれた部分の面積$S$を求めてみよう。
 \begin{ansBlockSize}{40}
  \RedBold{左右が分からないときは連立!}\\
  $y=6x-2x^2$と$x$軸($y=0$)を連立して
  \begin{align*}
   6x-2x^2 &= 0\\
   2x^2-6x &= 0\\
   x^2-3x &= 0\\
   x(x-3) &= 0\\
   x &= 0,\ 3
  \end{align*}
  グラフは右のようになるから\\
  $S = \displaystyle\int_{1}^{2}\left\{(6x-2x^2)-0\right\}\,dx$
 \end{ansBlockSize}
 
 \columnbreak{}
 
 \Rei{22}
 放物線$y=x^2-2x$と$x$軸で囲まれた部分の面積$S$を求めてみよう。
 \begin{ansBlockSize}{60}
  \RedBold{左右が分からないときは連立!}\\
  $y=x^2-2x$と$x$軸($y=0$)を連立して
  \begin{align*}
   x^2-2x &= 0\\
   x(x-2) &= 0\\
   x &= 0,\ 2
  \end{align*}
  グラフは右のようになるから\\
  $S = \displaystyle\int_{0}^{2}\left\{0-(x^2-2x)\right\}\,dx$
 \end{ansBlockSize}
 \vfill
 \null
 
 \columnbreak{}
 
 \Toi{34}
 次の放物線や直線で囲まれた部分の面積$S$を求めよ。
 \begin{enumerate}
  \item  放物線$y=x^2-3x-4$, $x$軸
	 \begin{ansBlockSize}{42}
	  \RedBold{左右が分からないときは連立!}\\
	  $y=x^2-3x-4$と$x$軸($y=0$)を連立して
	  \begin{align*}
	   x^2-3x-4 &= 0\\
	   (x+1)(x-4) &= 0\\
	   x &= -1,\ 4
	  \end{align*}
	  グラフは右のようになるから\\
	  $S = \displaystyle\int_{-1}^{4}\left\{0-(x^2-3x-4)\right\}\,dx$
	 \end{ansBlockSize}
	 \vfill
  \item  放物線$y=x^2-2$と$x$軸, $y$軸, 直線$x=1$
	 \begin{ansBlockSize}{42}
	  \RedBold{左右が分からないときは連立!}\\
	  $y=x^2-2$と$x$軸($y=0$)を連立して
	  \begin{align*}
	   x^2-2 &= 0\\
	   x^2 &= 2\\
	   x &= -\sqrt{2},\ \sqrt{2}
	  \end{align*}
	  グラフは右のようになるから\\
	  $S = \displaystyle\int_{0}^{1}\left\{0-(x^2-2)\right\}\,dx$
	 \end{ansBlockSize}
 \end{enumerate}
 
 \columnbreak{}
 \ReturnBox
 
\end{multicols*}
\end{document}

