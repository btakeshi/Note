% ============================================================================================================
\myHeader{第2章「2次関数」> 第3節「2次関数と方程式・不等式」・・・1回目}
% ============================================================================================================

% ------------------------------------------------------------------------------------------------------------
%
% ------------------------------------------------------------------------------------------------------------
\begin{multicols*}{4}
  % \textbook{Math1/Fig1_2024/Sec2/ks1_066.PNG}%============================================================066
  % \vfill
  % \Rei{8}%---------- p.066 例8 ----------
  % 次の2次方程式を解け。\vspace{.5zw}\\
  % (1)\ $2x^2+x-6=0$
  % \AnswerBox{5}
  % (2)\ $4x^2+20x+25=0$
  % \AnswerBox{5}
  % \columnbreak{}\\
  % \null\vfill
  % \Toi{19}
  % 次の2次方程式を解け。\vspace{.5zw}\\
  % (1)\ $x^2-2x-15=0$
  % \AnswerBox{5}
  % \vfill
  % (2)\ $3x^2+4x-4=0$
  % \AnswerBox{5}
  % \vfill
  % (3)\ $4x^2-12x+9=0$
  % \AnswerBox{5}
  % \vfill
  % (4)\ $3x=x^2$
  % \AnswerBox{5}
  % \columnbreak{}\\
  % \textbook{Math1/Fig1_2024/Sec2/ks1_067.PNG}%============================================================067
  % \columnbreak{}\\
  % \textbook{Math1/Fig1_2024/Sec2/ks1_068.PNG}%============================================================068
  % \vfill
  % \Rei{9}%---------- p.068 例9 ----------
  % 2次方程式$3x^2-5x+1=0$を解け。
  % \AnswerBox{11}
  % \columnbreak{}\\
  % \textbook{Math1/Fig1_2024/Sec2/ks1_068.PNG}%============================================================068
  % \vfill
  % \Rei{10}%---------- p.068 例10 ----------
  % 2次方程式$2x^2+6x+1=0$を解け。
  % \AnswerBox{11}
  % \columnbreak{}\\
  % \null\vfill
  % \Toi{20}
  % 次の2次方程式を解け。\vspace{.5zw}\\
  % (1)\ $x^2-x-3=0$
  % \AnswerBox{5}
  % \vfill
  % (2)\ $4x^2-4x+1=0$
  % \AnswerBox{5}
  % \vfill
  % (3)\ $x^2+3x+1=0$
  % \AnswerBox{5}
  % \vfill
  % (4)\ $x-1=2x^2-3$
  % \AnswerBox{5}
  % \columnbreak{}\\
  % \textbook{Math1/Fig1_2024/Sec2/ks1_069.PNG}%============================================================069
  % \null\vfill
  % \Toi{21}
  % 次の2次方程式を解け。\vspace{.5zw}\\
  % (1)\ $x^2+2x-5=0$
  % \AnswerBox{5}
  % \vfill
  % (2)\ $2x^2-4x+1=0$
  % \AnswerBox{5}
  % \columnbreak{}\\
  % \textbook{Math1/Fig1_2024/Sec2/ks1_070.PNG}%============================================================070
  % \vfill
  % \Rei{11}
  % 次の2次方程式の実数解の個数を調べよ。\vspace{.5zw}\\
  % (1)\ $2x^2-3x-1=0$
  % \AnswerBox{2.5}
  % \vfill
  % (2)\ $4x^2+12x+9=0$
  % \AnswerBox{2.5}
  % \vfill
  % (3)\ $6x^2-5x+2=0$
  % \AnswerBox{2.5}
  % \columnbreak{}\\
  \Toi{22}
  次の2次方程式の実数解の個数を調べよ。
  \vspace{.5zw}\\
  (1)\ $3x^2-2x-4=0$
  \AnswerBox{5}
  (2)\ $9x^2-6x+1=0$
  \AnswerBox{5}
  (3)\ $3x^2+5=0$
  \AnswerBox{5}
  (4)\ $(x+1)(x+2)=3$
  \AnswerBox{5}
  \columnbreak{}\\
  \textbook{Math1/Fig1_2024/Sec2/ks1_072.PNG}%============================================================072
  \columnbreak{}\\
  \textbook{Math1/Fig1_2024/Sec2/ks1_073.PNG}%============================================================073
  \columnbreak{}\\
  \Toi{26}
  次の2次関数のグラフと$x$軸の共有点は何個あるか調べよ。
  (1)\ $y=2x^2-3x-1$
  \AnswerBox{7}
  (2)\ $y=-4x^2+4x-1$
  \AnswerBox{7}
  (1)\ $y=3x^2+x+1$
  \AnswerBox{7}
  \columnbreak{}\\
  \textbook{Math1/Fig1_2024/Sec2/ks1_075.PNG}%============================================================075
  \columnbreak{}\\
  \textbook{Math1/Fig1_2024/Sec2/ks1_076.PNG}%============================================================076
  \columnbreak{}\\
  \null\vfill
  \Toi{27}
  次の2次不等式を解け。
  \vspace{.5zw}\\
  (1)\ $x^2-5x+6 \leqq 0$
  \AnswerBox{7}
  (2)\ $2x^2-7x-4 \geqq 0$
  \AnswerBox{7}
  (3)\ $x^2+x-4 > 0$
  \AnswerBox{7}
  \columnbreak{}\\
  \null\vfill
  (4)\ $2x^2-2x-3 \leqq 0$
  \AnswerBox{7}
  (5)\ $-x^2+6x-5 > 0$
  \AnswerBox{7}
  (6)\ $-x^2-5x-3 \leqq 0$
  \AnswerBox{7}
  \columnbreak{}\\
  \textbook{Math1/Fig1_2024/Sec2/ks1_077.PNG}%============================================================077
  \textbook{Math1/Fig1_2024/Sec2/ks1_078.PNG}%============================================================078
  \columnbreak{}\\
  \null\vfill
  \Toi{28}
  次の2次不等式を解け。
  \vspace{.5zw}\\
  (1)\ $x^2+10x+25 \geqq 0$
  \AnswerBox{5}
  (2)\ $9x^2-12x+4 \leqq 0$
  \AnswerBox{5}
  (3)\ $x^2+6x > -9$
  \AnswerBox{5}
  (4)\ $-x^2-4 > 4x$
  \AnswerBox{5}
  \columnbreak{}\\
  \null\vfill
  \Toi{29}
  次の2次不等式を解け。
  \vspace{.5zw}\\
  (1)\ $x^2-x+1 > 0$
  \AnswerBox{5}
  (2)\ $2x^2+x+1 < 0$
  \AnswerBox{5}
  (3)\ $4x^2+x \geqq -2$
  \AnswerBox{5}
  (4)\ $-3x^2+2x \geqq 1$
  \AnswerBox{5}
  \columnbreak{}\\
  \textbook{Math1/Fig1_2024/Sec2/ks1_078.PNG}%============================================================078
  \columnbreak{}\\
  \null\vfill
  \Toi{30}
  次の2次不等式を解け。
  \vspace{.5zw}\\
  (1)\ $x^2+2x > 0$
  \AnswerBox{7}
  (2)\ $x^2-4x-2 \leqq 0$
  \AnswerBox{7}
  (3)\ $4x-x^2 \geqq 4$
  \AnswerBox{7}
  \columnbreak{}\\
  \null\vfill
  (4)\ $4x^2-12x+9 > 0$
  \AnswerBox{7}
  (5)\ $-x^2-x-4 < 0$
  \AnswerBox{7}
  (6)\ $x^2+2x+3 \leqq 0$
  \AnswerBox{7}
  \columnbreak{}\\
  \null\vfill
  \Setumatu{4}
  次の2次不等式を解け。
  \vspace{.5zw}\\
  (1)\ $(x-1)(x+4) < 14$
  \AnswerBox{7}
  (2)\ $-x^2+x < \bunsuu14$
  \AnswerBox{7}
  \ReturnBox

  % \vfill
  % \Rei{6}%---------- p.061 例6 ----------
  % 2次関数$y=x^2-4x+5$($1 \leqq x \leqq 4$)の最大値,最小値があれば,それを求めよ。また,そのときの$x$の値も求めよ。
  % \AnswerBox{11}
  % \columnbreak{}\\
  % \textbook{Math1/Fig1_2024/Sec2/ks1_062.PNG}%============================================================062
  % \vfill
  % \Rei{7}%---------- p.062 例7 ----------
  % 2次関数$y=x^2-4x+5$($3 \leqq x \leqq 4$)の最大値,最小値があれば,それを求めよ。また,そのときの$x$の値も求めよ。
  % \AnswerBox{11}
  % \columnbreak{}\\
  % \textbook{Math1/Fig1_2024/Sec2/ks1_062.PNG}%============================================================062
  % \vfill
  % \Toi{15}%---------- p.062 問15 ----------
  % 次の2次関数の最大値,最小値があれば,それを求めよ。また,そのときの$x$の値を求めよ。\vspace{.5zw}\\
  % (1)\ $y=-2x^2+12x+2$($-1 \leqq x \leqq 2$)
  % \AnswerBox{10}
  % \columnbreak{}\\
  % \null\vfill
  % (2)\ $y=-x^2+6x+5$($1 \leqq x \leqq 4$)
  % \AnswerBox{11}
  % (3)\ $y=2x^2+4x+1$($0 \leqq x \leqq 3$)
  % \AnswerBox{11}
  % \columnbreak{}\\
  % \null\vfill
  % (4)\ $y=x^2-8x$($2 \leqq x \leqq 6$)
  % \AnswerBox{11}
  % (5)\ $y=-x^2-2x+3$($-2 < x < 1$)
  % \AnswerBox{11}
  % \columnbreak{}\\
  % \columnbreak{}\\
  % \textbook{Math1/Fig1_2024/Sec2/ks1_065.PNG}%============================================================062
  % \vfill
  % \Setumatu{1}%---------- p.065 節末1 ----------
  % 次の2次関数の最大値と最小値を求めよ。\vspace{.5zw}\\
  % (1)\ $y=-x^2+8x$($2 \leqq x \leqq 6$)
  % \AnswerBox{11}
  % \columnbreak{}\\
  % \null\vfill
  % (2)\ $y=3x^2+2x-1$($-2 \leqq x \leqq 0$)
  % \AnswerBox{11}
  % \vfill
  % (3)\ $y=2x^2-3x+3$($1 \leqq x \leqq 3$)
  % \AnswerBox{11}
  % \ReturnBox
  % % 次の関数$f(x)$に対して, $f(-1)$, $f(2)$, $f(a-1)$を求めよ。\\
  % % (1)\ $f(x)=4x-6$
  % % \AnswerBox{4.5}
  % % (2)\ $f(x)=x^2-1$
  % % \AnswerBox{4.5}
  % % \columnbreak\\
  % % \textbook{Math1/Fig1_2024/Sec2/ks1_047.PNG}
  % % \vfill
  % % \Toi{2}
  % % 周の長さ40 cmの長方形を作る。縦の長さを$x$ cm, 横の長さを$y$ cmとして, $y$を$x$の関数で表し, この関数の定義域と値域を求めよ。
  % % \AnswerBox{8}
  % % \columnbreak\\
  % % \textbook{Math1/Fig1_2024/Sec2/ks1_048.PNG}
  % % \Toi{3}
  % % 次のような座標をもつ点は第何象限にあるか。\\
  % % \begin{minipage}[t]{0.48\columnwidth}
  % %   (1)\ $(2,\ -3)$
  % %   \AnswerBox{5}
  % % \end{minipage}
  % % \begin{minipage}[t]{0.48\columnwidth}
  % %   (2)\ $(2,\ 3)$
  % %   \AnswerBox{5}
  % % \end{minipage}
  % % \\
  % % \begin{minipage}[t]{0.48\columnwidth}
  % %   (3)\ $(-2,\ 3)$
  % %   \AnswerBox{5}
  % % \end{minipage}
  % % \begin{minipage}[t]{0.48\columnwidth}
  % %   (4)\ $(-2,\ -3)$
  % %   \AnswerBox{5}
  % % \end{minipage}
  % % \columnbreak\\
  % % \textbook{Math1/Fig1_2024/Sec2/ks1_049.PNG}
  % % \Toi{4}
  % % 次の関数の最大値・最小値をグラフをかいて求めよ。\\
  % % (1)\ $y=x-5$($-2 \leqq x \leqq 2$)
  % % \AnswerBox{4.8}
  % % (2)\ $y=-2x+3$($1 \leqq x \leqq 3$)
  % % \AnswerBox{4.8}
\end{multicols*}
