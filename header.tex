\documentclass[landscape,a3paper,8pt]{ltjsarticle}
\usepackage{luatexja}


%% ----------------------------------------------------------------------------------------------------
%% 版面設定
%% ----------------------------------------------------------------------------------------------------
\usepackage[top=18mm,bottom=18mm,left=4mm,right=4mm]{geometry}
\setlength{\columnsep}{1.8\zw}
\setlength{\columnseprule}{0.1pt}
\parindent=0pt


%% ----------------------------------------------------------------------------------------------------
%% Tikz関係+色
%% ----------------------------------------------------------------------------------------------------
\usepackage{xcolor}
\usepackage{tikz}
\usepackage[most]{tcolorbox}


%% ----------------------------------------------------------------------------------------------------
%% ヘッダー・フッター
%% ----------------------------------------------------------------------------------------------------
\usepackage{fancyhdr}
\pagestyle{fancy}
\fancyhead[RO,RE]{\colorbox{blue!10!white}{※3回チェックします。1回目は授業後の取り組み状況(各2点)、2回目は宿題の取り組み状況(各2点)、3回目は最終評価(各2点)です。}}
\fancyfoot[RO,RE]{\textcolor{blue}{※青い枠が課題です。青い枠内で解いてください。} \textcolor{red}{※赤い枠も課題です。こちらは記述問題としてチェックします。途中もしっかりと残しておこう。} ※少しずつ〇付けをしていきます。1週間ぐらいしたら評価を行います。お早めに。}
\fancyfoot[CO,CE]{}
\newcommand{\myHeader}[1]{\fancyhead[LO,LE]{\colorbox{red!10!white}{\large\gtfamily\bfseries{}#1}}}


%% ----------------------------------------------------------------------------------------------------
%% 環境作成 by environ
%% ----------------------------------------------------------------------------------------------------
\usepackage{environ}
%% --------------------------------------------------------------------------------
%% ---------- \BODY で高さを自動調整する解答ボックス
\NewEnviron{ansBlockAuto}{%
 \begin{tcolorbox}[colframe=blue,colback=white]
  \BODY
 \end{tcolorbox}
}
%% ---------- 高さを指定した解答ボックス
\NewEnviron{ansBlockSize}[1]{%
 \begin{tcolorbox}[colframe=blue,colback=white,height=#1\zw]
  \BODY
 \end{tcolorbox}
}
%% ---------- 公式などを目立たせるボックス
\NewEnviron{thmBlock}{%
 \begin{tcolorbox}[colframe=red,colback=red!5!white]
  \BODY
 \end{tcolorbox}
}


%% ----------------------------------------------------------------------------------------------------
%% コマンド作成
%% ----------------------------------------------------------------------------------------------------
%% --------------------------------------------------------------------------------------教科書取り込み
\newcommand{\textbook}[2][1.00]{%
  \begin{minipage}{0.9\columnwidth}
    \includegraphics[trim=50 100 110 100, width=#1\textwidth, clip]{#2}
  \end{minipage}
}
\newcommand{\AnswerBox}[2][1.00]{%
  \vspace{.5zw}\\
  \begin{tikzpicture}%
    \draw[dotted,very thick,blue] (0,0) rectangle (#1\columnwidth,#2\zw);%
  \end{tikzpicture}
}
\newcommand{\ReturnBox}{%
    おつかれさまでした。\vspace{1\zw}\\
    \textcolor{blue}{\bfseries すべて取り組んだら今回の学びを「振り返り」ましょう。}\\
    \qquad{}・評価2点・・・記入できた。\\
    \qquad{}・評価2点・・・文章で書けた。\\
    \qquad{}・評価2点・・・2文以上書けた。\vspace{1\zw}\\
    \textcolor{red}{\bfseries Y.やったこと}\vspace{-.5\zw}\\
    {\footnotesize{}今回の課題で学んだことを自分のことばでまとめよう}\vspace{.5\zw}\\
    \begin{tikzpicture}
      \draw[ultra thick,red] (0,0) rectangle (0.95\columnwidth,5);
    \end{tikzpicture}
    \vspace{1\zw}
    \textcolor{red}{\bfseries W.わかったこと}\vspace{-.5\zw}\\
    {\footnotesize{}今回の課題で理解したことを自分のことばでまとめよう}\vspace{.5\zw}\\
    \begin{tikzpicture}
      \draw[ultra thick,red] (0,0) rectangle (0.95\columnwidth,5);
    \end{tikzpicture}
    \vspace{1\zw}
    \textcolor{red}{\bfseries T.次にやること}\vspace{-.5\zw}\\
    {\footnotesize{}次の課題に向けて何をすべきか自分のことばで残しましょう}\vspace{.5\zw}\\
    \begin{tikzpicture}
      \draw[ultra thick,red] (0,0) rectangle (0.95\columnwidth,5);
    \end{tikzpicture}
    \vfill
}



\usepackage{graphicx}%includegraphics
\usepackage{multicol}




%% ----------------------------------------------------------------------------------------------------
%% 数学関係
%% ----------------------------------------------------------------------------------------------------
\usepackage{emath}



\newcommand{\Rei}[1]{\fcolorbox{blue}{cyan!20!white}{\ 例#1\ }\vspace{.5\zw}\\}
\newcommand{\Reidai}[1]{\fcolorbox{green}{green!20!white}{\ 例題#1\ }\quad{}}
\newcommand{\Toi}[1]{\fcolorbox{red}{pink!20!white}{\ 問#1\ }\vspace{.5\zw}\\}
\newcommand{\Setumatu}[1]{\fcolorbox{black}{gray!20!white}{\ 節末#1\ }\quad{}}
\newcommand{\Libry}[1]{\fcolorbox{orange}{orange!20!white}{\ #1\ }\vspace{.5\zw}\\}

\newcommand{\RedBold}[1]{\textcolor{red}{\textbf{#1}}}

\renewcommand{\labelenumi}{(\arabic{enumi}).\ }